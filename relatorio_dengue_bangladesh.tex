% Options for packages loaded elsewhere
\PassOptionsToPackage{unicode}{hyperref}
\PassOptionsToPackage{hyphens}{url}
%
\documentclass[
]{article}
\usepackage{amsmath,amssymb}
\usepackage{iftex}
\ifPDFTeX
  \usepackage[T1]{fontenc}
  \usepackage[utf8]{inputenc}
  \usepackage{textcomp} % provide euro and other symbols
\else % if luatex or xetex
  \usepackage{unicode-math} % this also loads fontspec
  \defaultfontfeatures{Scale=MatchLowercase}
  \defaultfontfeatures[\rmfamily]{Ligatures=TeX,Scale=1}
\fi
\usepackage{lmodern}
\ifPDFTeX\else
  % xetex/luatex font selection
\fi
% Use upquote if available, for straight quotes in verbatim environments
\IfFileExists{upquote.sty}{\usepackage{upquote}}{}
\IfFileExists{microtype.sty}{% use microtype if available
  \usepackage[]{microtype}
  \UseMicrotypeSet[protrusion]{basicmath} % disable protrusion for tt fonts
}{}
\makeatletter
\@ifundefined{KOMAClassName}{% if non-KOMA class
  \IfFileExists{parskip.sty}{%
    \usepackage{parskip}
  }{% else
    \setlength{\parindent}{0pt}
    \setlength{\parskip}{6pt plus 2pt minus 1pt}}
}{% if KOMA class
  \KOMAoptions{parskip=half}}
\makeatother
\usepackage{xcolor}
\usepackage[margin=1in]{geometry}
\usepackage{color}
\usepackage{fancyvrb}
\newcommand{\VerbBar}{|}
\newcommand{\VERB}{\Verb[commandchars=\\\{\}]}
\DefineVerbatimEnvironment{Highlighting}{Verbatim}{commandchars=\\\{\}}
% Add ',fontsize=\small' for more characters per line
\usepackage{framed}
\definecolor{shadecolor}{RGB}{248,248,248}
\newenvironment{Shaded}{\begin{snugshade}}{\end{snugshade}}
\newcommand{\AlertTok}[1]{\textcolor[rgb]{0.94,0.16,0.16}{#1}}
\newcommand{\AnnotationTok}[1]{\textcolor[rgb]{0.56,0.35,0.01}{\textbf{\textit{#1}}}}
\newcommand{\AttributeTok}[1]{\textcolor[rgb]{0.13,0.29,0.53}{#1}}
\newcommand{\BaseNTok}[1]{\textcolor[rgb]{0.00,0.00,0.81}{#1}}
\newcommand{\BuiltInTok}[1]{#1}
\newcommand{\CharTok}[1]{\textcolor[rgb]{0.31,0.60,0.02}{#1}}
\newcommand{\CommentTok}[1]{\textcolor[rgb]{0.56,0.35,0.01}{\textit{#1}}}
\newcommand{\CommentVarTok}[1]{\textcolor[rgb]{0.56,0.35,0.01}{\textbf{\textit{#1}}}}
\newcommand{\ConstantTok}[1]{\textcolor[rgb]{0.56,0.35,0.01}{#1}}
\newcommand{\ControlFlowTok}[1]{\textcolor[rgb]{0.13,0.29,0.53}{\textbf{#1}}}
\newcommand{\DataTypeTok}[1]{\textcolor[rgb]{0.13,0.29,0.53}{#1}}
\newcommand{\DecValTok}[1]{\textcolor[rgb]{0.00,0.00,0.81}{#1}}
\newcommand{\DocumentationTok}[1]{\textcolor[rgb]{0.56,0.35,0.01}{\textbf{\textit{#1}}}}
\newcommand{\ErrorTok}[1]{\textcolor[rgb]{0.64,0.00,0.00}{\textbf{#1}}}
\newcommand{\ExtensionTok}[1]{#1}
\newcommand{\FloatTok}[1]{\textcolor[rgb]{0.00,0.00,0.81}{#1}}
\newcommand{\FunctionTok}[1]{\textcolor[rgb]{0.13,0.29,0.53}{\textbf{#1}}}
\newcommand{\ImportTok}[1]{#1}
\newcommand{\InformationTok}[1]{\textcolor[rgb]{0.56,0.35,0.01}{\textbf{\textit{#1}}}}
\newcommand{\KeywordTok}[1]{\textcolor[rgb]{0.13,0.29,0.53}{\textbf{#1}}}
\newcommand{\NormalTok}[1]{#1}
\newcommand{\OperatorTok}[1]{\textcolor[rgb]{0.81,0.36,0.00}{\textbf{#1}}}
\newcommand{\OtherTok}[1]{\textcolor[rgb]{0.56,0.35,0.01}{#1}}
\newcommand{\PreprocessorTok}[1]{\textcolor[rgb]{0.56,0.35,0.01}{\textit{#1}}}
\newcommand{\RegionMarkerTok}[1]{#1}
\newcommand{\SpecialCharTok}[1]{\textcolor[rgb]{0.81,0.36,0.00}{\textbf{#1}}}
\newcommand{\SpecialStringTok}[1]{\textcolor[rgb]{0.31,0.60,0.02}{#1}}
\newcommand{\StringTok}[1]{\textcolor[rgb]{0.31,0.60,0.02}{#1}}
\newcommand{\VariableTok}[1]{\textcolor[rgb]{0.00,0.00,0.00}{#1}}
\newcommand{\VerbatimStringTok}[1]{\textcolor[rgb]{0.31,0.60,0.02}{#1}}
\newcommand{\WarningTok}[1]{\textcolor[rgb]{0.56,0.35,0.01}{\textbf{\textit{#1}}}}
\usepackage{longtable,booktabs,array}
\usepackage{calc} % for calculating minipage widths
% Correct order of tables after \paragraph or \subparagraph
\usepackage{etoolbox}
\makeatletter
\patchcmd\longtable{\par}{\if@noskipsec\mbox{}\fi\par}{}{}
\makeatother
% Allow footnotes in longtable head/foot
\IfFileExists{footnotehyper.sty}{\usepackage{footnotehyper}}{\usepackage{footnote}}
\makesavenoteenv{longtable}
\usepackage{graphicx}
\makeatletter
\def\maxwidth{\ifdim\Gin@nat@width>\linewidth\linewidth\else\Gin@nat@width\fi}
\def\maxheight{\ifdim\Gin@nat@height>\textheight\textheight\else\Gin@nat@height\fi}
\makeatother
% Scale images if necessary, so that they will not overflow the page
% margins by default, and it is still possible to overwrite the defaults
% using explicit options in \includegraphics[width, height, ...]{}
\setkeys{Gin}{width=\maxwidth,height=\maxheight,keepaspectratio}
% Set default figure placement to htbp
\makeatletter
\def\fps@figure{htbp}
\makeatother
\setlength{\emergencystretch}{3em} % prevent overfull lines
\providecommand{\tightlist}{%
  \setlength{\itemsep}{0pt}\setlength{\parskip}{0pt}}
\setcounter{secnumdepth}{-\maxdimen} % remove section numbering
\ifLuaTeX
  \usepackage{selnolig}  % disable illegal ligatures
\fi
\IfFileExists{bookmark.sty}{\usepackage{bookmark}}{\usepackage{hyperref}}
\IfFileExists{xurl.sty}{\usepackage{xurl}}{} % add URL line breaks if available
\urlstyle{same}
\hypersetup{
  pdftitle={Relatório de Dengue em Bangladesh},
  pdfauthor={MarianaFreitas e Aline Pires},
  hidelinks,
  pdfcreator={LaTeX via pandoc}}

\title{Relatório de Dengue em Bangladesh}
\author{MarianaFreitas e Aline Pires}
\date{}

\begin{document}
\maketitle

\hypertarget{introduuxe7uxe3o}{%
\section{Introdução}\label{introduuxe7uxe3o}}

\hypertarget{metodologia}{%
\section{Metodologia}\label{metodologia}}

Dados categorizados são conjuntos de dados cujas variáveis são
categóricas, ou seja, representam característica, qualidade ou atributo.
Essas variáveis categóricas podem ser nominais, quando as classes da
variável não tem ordem natural (gênero, tipo sanguíneo..) ou ordinais,
quando as classes apresentam ordem natural (nível de escolaridade, grau
de dor,\ldots). Uma importante técnica na análise de dados categorizados
são as tabelas de contigência, ideais para organizar a frequência das
interseções entre as variáveis categóricas, permitindo a observação de
associação entre variáveis, cálculo de medidas de desempenho de testes
diagnósticos e realização de testes de associação, simetria ou
homogeneidade entre variáveis.

No contexto de testes diagnósticos - testes que identificam se um
indivíduo apresenta ou não determinada doença ou condição - é possível
calcular algumas medidas para aavaliar a performance dos testes, já que
estão sujeitos a erros e, consequentemente, seus resultados apresentam
grau de incerteza. Duas medidas muito utilizadas são a sensibilidade e
especificidade. A sensibilidade é calculada como a razão de verdadeiros
positivos (doentes cujo teste foi positivo) em relação à soma de falsos
negativos (doentes cujo teste foi negativo) e verdadeiros positivos. Já
a especificidade corresponde à razão de verdadeiros negativos (não
doentes cujo teste foi negativo) em relação à soma de verdadeiros
negativos e falsos positivos (não doentes cujo teste foi positivo).
Assim, a sensibilidade pode ser interpretada como a probabilidade do
teste ser positivo dados que o indivíduo está doente e a especificidade
como a probabilidade do teste ser negativo dado que o indivíduo não está
doente. Um bom teste apresenta sensibilidade e especificidade altas,
sendo que podem variar de 0 a 1.

O tipo de estudo e delineamento amostral são importantes para a
interpretação de resultados. Nesse caso, será justificado que se trata
de um estudo transversal, mas não há informações suficientes para
definir o delineamento amostral. Para esse tipo de estudo, cabe
verificar se há diferenças nas proporções de doentes em diferentes
classes de variáveis binárias. Para isso, é calculada a estimação
pontual da diferença entre as porporções e em seguida é feito o
intervalo de confiança. Se o intervalo de confiança incluir zero, não se
pode afirmar que há uma diferença entre as proporções. Caso contrário,
há diferença nas porporções a uma determinada nível de confiança - nesse
trabalho foi utilizado 95\%. Outra importante alternativa para avaliar
associação em tabelas de contigência \(2x2\) são as razão de chances,
medida apropriada para o tipo de estudo transversal. A razão de chances
se trata da razão entre a chance de uma classe apresentar a doença e a
chance da outra classe apresentar a doença. Foi feita uma estimativa da
razão de chance e em seguida foi complementada por inferência
estatística, com cálculo do logaritmo da razão de chances e respectivos
intervalos de confiança obtidos por aproximação normal. Se o intervalo
de confiança incluir 1, não se pode afirmar que há diferença entre as
chances, caso contrário há diferença a um determinado nível de
confiança.

Além de inferência estatística para as proporções, também foram feitos
testes para avaliar independência, associação, simetria e homogeneidade.
Primeiro forma feitos testes específicoa para variáveis ordinais. Para
verificar a intensidade e direção da associação entre variáveis
ordinais, foram aplicadas os testes Gama de Goodman e Kruskal, Tau de
Kendall e Tau-b de Kendall - que consideram a ordenação das classes. A
Gama de Goodman e Kruskal se baseia em pares concordantes e discordantes
em tabelas de contingência, variando de -1 a 1, indicando associação
perfeita negativa ou positiva, respectivamente, e desconsidera os pares
empatados. As medidas Tau de Kendall e Tau-b de Kendall corrigem a Gama
ao considerar empates nas margens. O Tau-b é útil para tabelas não
quadradas, visto que ajusta a estatística levando em conta o número de
empates nas linhas e colunas. Para testar tendência linear entre
variáveis ordinais, podem ser usados os testes de Cochran-Armitage - em
tabelas para verificar se a proporção de sucesso aumenta ou diminui
linearmente com as categorias das variáveis ordinais - e o teste de
Mantel - em tabelas \(sxr\), avaliando a presença de uma tendência
linear global entre variáveis ordinais. Aqui foi aplicado apenas o
primeiro teste, já que há apenas duas variáveis ordinais.

Para analisar a associação entre duas variáveis categóricas controlando
por uma terceira variável, foram construídas tabelas de contingência
parciais, e calculadas razões de chances condicionais em cada classe,
permitindo avaliar se a associação é condicionalmente homogênea entre as
classes. A homogeneidade das razões de chances foi testada com o teste
de Breslow-Day, que avalia se as ORs são estatisticamente iguais entre
as classes. Quando a homogeneidade foi aceita, foi utilizado o teste de
Mantel-Haenszel, que fornece uma razão de chances combinada ajustada,
além de um teste de associação global. Essa técinica é importante para
lidar com casos em que ocorre o Paradoxo de Simpson - quando a
associação entre duas variáveis muda após o controle por uma terceira.

Para tabelas de contingência com dimensões \(r×s\), foram aplicados
algins testes para avaliar associação e simetria. O teste qui-quadrado
de Pearson foi utilizado para verificar a independência entre linhas e
colunas. O teste de razão de verossimilhança, uma alternativa ao
qui-quadrado, tem base no modelo de log-verossimilhança, sendo mais
adequado em amostras pequenas ou quando os pressupostos de normalidade
não são satisfeitos. O teste de homogeneidade foi aplicado em situações
nas quais uma das variáveis representa grupos e a outra representa
categorias de resposta, buscando verificar se a distribuição de
respostas é homogênea entre os grupos. O teste de simetria foi utilizado
em tabelas quadradas para avaliar se a frequência de observações na
célula \((i,j)\) é igual a \((j, i)\), útil para dados pareados ou
classificações duplas. O teste de homogeneidade marginal, também em
tabelas quadradas, verificou se as distribuições marginais das linhas e
colunas são idênticas, independentemente da simetria.

Por fim, foi abordada a etapa de modelagem em tabelas de contigência.
Foi ajustado um modelo de regressão logística, permitindo estimar a
probabilidade de doença como função das variáveis explicativas
categóricas. Os coeficientes do modelo foram interpretados em termos do
log das razões de chance. Pela base de dados desse trabalho se tratar de
um problema com variável resposta binária, os modelos log-lineares não
foram utilizados. Fio feita uma seleção do modelo de regressão logística
mais adequado a partir do critério de Informação Akaike (AIC) e análise
do ajuste do modelo.

Todas as análises foram feitas utilizando o software R, com pacotes
específicos mencionados ao longo do relatório.

\hypertarget{resultados}{%
\section{Resultados}\label{resultados}}

\hypertarget{anuxe1lise-exploratuxf3ria}{%
\subsection{Análise exploratória}\label{anuxe1lise-exploratuxf3ria}}

\begin{itemize}
\tightlist
\item
  Falar sobre os dados (variáveis, fonte, \ldots) Aline
\item
  falar que apenas age\_less\_16 e AreaType são ordinais Aline
\item
  Fazer análises que julgar interessantes Aline
\end{itemize}

\hypertarget{avaliauxe7uxe3o-de-testes-diagnuxf3sticos}{%
\subsection{Avaliação de Testes
Diagnósticos}\label{avaliauxe7uxe3o-de-testes-diagnuxf3sticos}}

O conjunto de dados apresenta três testes para detectar a dengue:
denotados por NS1, IgG e IgM. Considerando apenas as informações sobre
os resultados desses três testes e a presença ou não da doença nos
indivíduos testados, são obtidas as tabelas 1, 2 e 3 para NS1, IgG e
IgM, respectivamente.

Tabela 1: Distribuição dos desfechos segundo teste NS1.

\begin{longtable}[]{@{}lll@{}}
\toprule\noalign{}
NS1/Outcome & Negative & Positive \\
\midrule\noalign{}
\endhead
\bottomrule\noalign{}
\endlastfoot
Negative & 467 & 14 \\
Positive & 0 & 519 \\
\end{longtable}

Tabela 2: Distribuição dos desfechos segundo teste IgG.

\begin{longtable}[]{@{}lll@{}}
\toprule\noalign{}
IgG/Outcome & Negative & Positive \\
\midrule\noalign{}
\endhead
\bottomrule\noalign{}
\endlastfoot
Negative & 467 & 0 \\
Positive & 0 & 533 \\
\end{longtable}

Tabela 3: Distribuição dos desfechos segundo teste IgM.

\begin{longtable}[]{@{}lll@{}}
\toprule\noalign{}
IgM/Outcome & Negative & Positive \\
\midrule\noalign{}
\endhead
\bottomrule\noalign{}
\endlastfoot
Negative & 251 & 274 \\
Positive & 216 & 259 \\
\end{longtable}

Apenas observando as tabelas parciais é possível notar que o teste mais
preciso parece ser o IgG, enquanto o de menor eficácia seria o IgM. No
entanto, essa intuição pode ser formalizada utilizando as medidas de
sensibilidade e especificidade apresentadas na tabela 4.

Tabela 4: Medidas de avaliação dos testes diagnósticos.

\begin{longtable}[]{@{}lll@{}}
\toprule\noalign{}
Teste Diagnóstico & Sensibilidade & Especificidade \\
\midrule\noalign{}
\endhead
\bottomrule\noalign{}
\endlastfoot
NS1 & 0,974 & 1 \\
IgG & 1 & 1 \\
IgM & 0.486 & 0.538 \\
\end{longtable}

Concluões sobre os testes: + O teste NS1 apresentou boa performance no
geral, classificando corretamente todos os que não tinham dengue e
também com alta sensibilidade - indicando que classificou grande parte
dos indivíduos com a diença corretamente. + O teste IgG classificou
corretamente todos os indivíduos. + O teste IgM teve performance bem
ruim, com ambas as medidas baixas.

\hypertarget{tabelas-de-contiguxeancia}{%
\subsection{Tabelas de Contigência}\label{tabelas-de-contiguxeancia}}

Para decidir que ferramentas serão usadas na análise de uma tabela de
contigência, é importante antes de qualquer coisa entender qual tipo é o
tipo de estudo. A partir das informações fornecidas pela fonte dos
dados, é possível inferir que se trata de um estudo transversal, pois os
dados são colhidos em um ponto específico no tempo após a ocorrência ou
não ocorrãncia de dengue, não houve nenhum tipo de intervenção ou
acompanhamento dos indivíduos. Esse tipo de estudo permite a verificar
se há diferenças nas proporções de doentes nas classes de variáveis
binárias, a partir da criação de tabelas \(2x2\) considerando a variável
explicativa binária de interesse e a variável resposta. Nas tabelas 5, 6
e 7 são mostradas as tabelas \(2x2\) para as variáveis binárias
\texttt{Gender}, \texttt{AreaType} e \texttt{Age\_less\_16\_years}.

Tabela 5: Distribuição dos desfechos segundo gênero.

\begin{longtable}[]{@{}lll@{}}
\toprule\noalign{}
Gender/Outcome & Negative & Positive \\
\midrule\noalign{}
\endhead
\bottomrule\noalign{}
\endlastfoot
Female & 243 & 281 \\
Male & 224 & 252 \\
\end{longtable}

Tabela 6: Distribuição dos desfechos segundo tipo de área.

\begin{longtable}[]{@{}lll@{}}
\toprule\noalign{}
AreaType/Outcome & Negative & Positive \\
\midrule\noalign{}
\endhead
\bottomrule\noalign{}
\endlastfoot
Developed & 244 & 257 \\
Undeveloped & 223 & 276 \\
\end{longtable}

Tabela 7: Distribuição dos desfechos segundo classificação da idade.

\begin{longtable}[]{@{}lll@{}}
\toprule\noalign{}
Age\_less\_16\_years/Outcome & Negative & Positive \\
\midrule\noalign{}
\endhead
\bottomrule\noalign{}
\endlastfoot
No & 406 & 452 \\
Yes & 61 & 81 \\
\end{longtable}

Inicialmente não é possível concluir muito apenas observando as tabelas,
então é útil calcular as estimações pontuais e intervalos de confiança
para as diferenças de proporção de doentes entre as classes de cada
variável. Os resultados estão apresentados na tabela 8.

Tabela 8: Resultados para diferenças de proporções.

\begin{longtable}[]{@{}
  >{\raggedright\arraybackslash}p{(\columnwidth - 6\tabcolsep) * \real{0.2065}}
  >{\raggedright\arraybackslash}p{(\columnwidth - 6\tabcolsep) * \real{0.3804}}
  >{\raggedright\arraybackslash}p{(\columnwidth - 6\tabcolsep) * \real{0.2065}}
  >{\raggedright\arraybackslash}p{(\columnwidth - 6\tabcolsep) * \real{0.2065}}@{}}
\toprule\noalign{}
\begin{minipage}[b]{\linewidth}\raggedright
Variável
\end{minipage} & \begin{minipage}[b]{\linewidth}\raggedright
Diferença na proporção de doentes
\end{minipage} & \begin{minipage}[b]{\linewidth}\raggedright
Estimação pontual
\end{minipage} & \begin{minipage}[b]{\linewidth}\raggedright
IC (95\%)
\end{minipage} \\
\midrule\noalign{}
\endhead
\bottomrule\noalign{}
\endlastfoot
Gender & Female - Male & 0.006848 & {[}-0.055, 0.069{]} \\
AreaType & Undeveloped - Developed & 0.040132 & {[}-0.021, 0.102{]} \\
Age\_less\_16\_years & No - Yes & -0.04361 & {[}-0.132, 0.044{]} \\
\end{longtable}

Todos os intervalos de confiança, a um nível de 95\% contém o valor 0.
Dessa forma, assumimos que não há diferença nas proporções de doentes
entre as classes das variáveis \texttt{Gender}, \texttt{AreaType} e
\texttt{Age\_less\_16\_years}. As estimações pontuais também foram bem
próximas de zero, o que reafirma que as diferenças são fruto da
aleatoriedade, não das classes.

Uma medida importante para verificar se há associação entre a ocorrência
de doença e as variáveis binárias mencionadas é a razão de chance. suas
estimativas pontuais e intervalos de confiança calculados a partir da
exponencial dos logs das razões de chance constam na tabela 9.

Tabela 9: Resultados para razões de chances.

\begin{longtable}[]{@{}
  >{\raggedright\arraybackslash}p{(\columnwidth - 6\tabcolsep) * \real{0.1959}}
  >{\raggedright\arraybackslash}p{(\columnwidth - 6\tabcolsep) * \real{0.4124}}
  >{\raggedright\arraybackslash}p{(\columnwidth - 6\tabcolsep) * \real{0.1959}}
  >{\raggedright\arraybackslash}p{(\columnwidth - 6\tabcolsep) * \real{0.1959}}@{}}
\toprule\noalign{}
\begin{minipage}[b]{\linewidth}\raggedright
Variável
\end{minipage} & \begin{minipage}[b]{\linewidth}\raggedright
Razão das chances de apresentar dengue
\end{minipage} & \begin{minipage}[b]{\linewidth}\raggedright
Estimação pontual
\end{minipage} & \begin{minipage}[b]{\linewidth}\raggedright
IC (95\%)
\end{minipage} \\
\midrule\noalign{}
\endhead
\bottomrule\noalign{}
\endlastfoot
Gender & Male/Female & 0.973 & {[} 0.759 , 1.248 {]} \\
AreaType & Developed/Undeveloped & 0.851 & {[} 0.664 , 1.091 {]} \\
Age\_less\_16\_years & Yes/No & 1.19 & {[} 0.834 , 1.707 {]} \\
\end{longtable}

Visto que todos os intervalos de confiança para a razão de chance contém
1, pode-se afirmar que não há diferença causada pelas classes entre as
chances de ter dengue. As estimações pontuais também estão bem próximas
de 1, reforçando essa ideia.

\hypertarget{inferuxeancia-para-tabelas-de-contiguxeancia}{%
\subsection{Inferência para Tabelas de
Contigência}\label{inferuxeancia-para-tabelas-de-contiguxeancia}}

\begin{itemize}
\tightlist
\item
  Teste Gama de Goodman e Kruskal (usar variáveis idade e areatype)
  Aline
\item
  Teste Tau de Kendall e Tau-b de Kendall (usar variáveis idade e
  areatype) Aline
\item
  Cochran-Armitage para tendência linear (usar variáveis idade e
  areatype) Aline
\end{itemize}

\hypertarget{associauxe7uxe3o-em-tabelas-de-contiguxeancia}{%
\subsection{Associação em Tabelas de
Contigência}\label{associauxe7uxe3o-em-tabelas-de-contiguxeancia}}

\begin{itemize}
\tightlist
\item
  Teste de Breslow Day Aline
\item
  Se homogeneidade aceita: teste de Mantel-Haenszel - falar se ocorre ou
  não o paradoxo de simpason Aline
\item
  Teste qui-quadrado Aline
\item
  Teste de razão de verossimilhança Aline
\item
  Teste de homogeneidade Aline
\item
  Teste de simetria (testar simetria entre IgG, IgM, e NS1? ou de outras
  variáveis?) Aline
\item
  Teste de homogeneidade marginal Aline
\end{itemize}

\hypertarget{regressuxe3o-loguxedstica}{%
\subsection{Regressão Logística}\label{regressuxe3o-loguxedstica}}

\begin{itemize}
\tightlist
\item
  Testar opções de modelos de regressão logística Mariana
\item
  Selecionar baseado no AIC Mariana
\item
  Interpretação do modelo Mariana
\end{itemize}

\hypertarget{conclusuxe3o}{%
\section{Conclusão}\label{conclusuxe3o}}

\hypertarget{apuxeandice}{%
\section{Apêndice}\label{apuxeandice}}

Códigos utilizados em R.

\begin{Shaded}
\begin{Highlighting}[]
\CommentTok{\# Pacotes utilizados}
\FunctionTok{library}\NormalTok{(readr)}
\FunctionTok{library}\NormalTok{(dplyr)}
\FunctionTok{library}\NormalTok{(janitor)}
\FunctionTok{library}\NormalTok{(tidyr)}
\FunctionTok{library}\NormalTok{(maxstat)}
\FunctionTok{library}\NormalTok{(data.table)}
\FunctionTok{library}\NormalTok{(caret)}
\FunctionTok{library}\NormalTok{(PropCIs)}

\CommentTok{\# Carregando dados }
\NormalTok{dados\_aux }\OtherTok{\textless{}{-}} \FunctionTok{fread}\NormalTok{(}\StringTok{"https://www.kaggle.com/api/v1/datasets/download/kawsarahmad/dengue{-}dataset{-}bangladesh"}\NormalTok{)}

\CommentTok{\# Pontos de corte para a idade}
\NormalTok{cutpoint }\OtherTok{\textless{}{-}} \FunctionTok{maxstat.test}\NormalTok{(Outcome }\SpecialCharTok{\textasciitilde{}}\NormalTok{ Age, }\AttributeTok{data =}\NormalTok{ dados\_aux)}
\NormalTok{cutpoint}\SpecialCharTok{$}\NormalTok{estimate}


\NormalTok{dados }\OtherTok{\textless{}{-}}\NormalTok{ dados\_aux }\SpecialCharTok{|\textgreater{}}
  \FunctionTok{mutate}\NormalTok{(}
    \AttributeTok{Age\_less\_16\_years =} \FunctionTok{case\_when}\NormalTok{(Age }\SpecialCharTok{\textless{}} \DecValTok{16} \SpecialCharTok{\textasciitilde{}} \StringTok{"Yes"}\NormalTok{,}
\NormalTok{                                    Age }\SpecialCharTok{\textgreater{}=} \DecValTok{16} \SpecialCharTok{\textasciitilde{}} \StringTok{"No"}\NormalTok{),}
    \AttributeTok{NS1 =} \FunctionTok{case\_when}\NormalTok{(NS1 }\SpecialCharTok{==} \DecValTok{1} \SpecialCharTok{\textasciitilde{}} \StringTok{"Positive"}\NormalTok{,}
\NormalTok{                    NS1 }\SpecialCharTok{==} \DecValTok{0} \SpecialCharTok{\textasciitilde{}} \StringTok{"Negative"}\NormalTok{),}
    \AttributeTok{IgG =} \FunctionTok{case\_when}\NormalTok{(IgG }\SpecialCharTok{==} \DecValTok{1} \SpecialCharTok{\textasciitilde{}} \StringTok{"Positive"}\NormalTok{,}
\NormalTok{                    IgG }\SpecialCharTok{==} \DecValTok{0} \SpecialCharTok{\textasciitilde{}} \StringTok{"Negative"}\NormalTok{),}
    \AttributeTok{IgM =} \FunctionTok{case\_when}\NormalTok{(IgM }\SpecialCharTok{==} \DecValTok{1} \SpecialCharTok{\textasciitilde{}} \StringTok{"Positive"}\NormalTok{,}
\NormalTok{                    IgM }\SpecialCharTok{==} \DecValTok{0} \SpecialCharTok{\textasciitilde{}} \StringTok{"Negative"}\NormalTok{),}
    \AttributeTok{Outcome =} \FunctionTok{case\_when}\NormalTok{(Outcome }\SpecialCharTok{==} \DecValTok{1} \SpecialCharTok{\textasciitilde{}} \StringTok{"Positive"}\NormalTok{,}
\NormalTok{                    Outcome }\SpecialCharTok{==} \DecValTok{0} \SpecialCharTok{\textasciitilde{}} \StringTok{"Negative"}\NormalTok{)}
\NormalTok{  ) }\SpecialCharTok{|\textgreater{}}
  \FunctionTok{select}\NormalTok{(}\SpecialCharTok{{-}}\FunctionTok{c}\NormalTok{(Age, District))}

\CommentTok{\# Número de variáveis faltantes}
\FunctionTok{sum}\NormalTok{(}\FunctionTok{is.na}\NormalTok{(dados))}

\CommentTok{\# Transformando todas as variáveis em fator}

\NormalTok{dados }\OtherTok{\textless{}{-}}\NormalTok{ dados }\SpecialCharTok{|\textgreater{}}
  \FunctionTok{mutate}\NormalTok{(}
    \AttributeTok{Gender =} \FunctionTok{factor}\NormalTok{(Gender),}
    \AttributeTok{IgG =} \FunctionTok{factor}\NormalTok{(IgG),}
    \AttributeTok{NS1 =} \FunctionTok{factor}\NormalTok{(NS1),}
    \AttributeTok{IgM =} \FunctionTok{factor}\NormalTok{(IgM),}
    \AttributeTok{Area =} \FunctionTok{factor}\NormalTok{(Area),}
    \AttributeTok{AreaType =} \FunctionTok{factor}\NormalTok{(AreaType, }\AttributeTok{levels =} \FunctionTok{c}\NormalTok{(}\StringTok{"Undeveloped"}\NormalTok{, }\StringTok{"Developed"}\NormalTok{)),}
    \AttributeTok{HouseType =} \FunctionTok{factor}\NormalTok{(HouseType, }\AttributeTok{levels =} \FunctionTok{c}\NormalTok{(}\StringTok{"Tinshed"}\NormalTok{, }\StringTok{"Building"}\NormalTok{, }\StringTok{"Other"}\NormalTok{)),}
    \AttributeTok{Age\_less\_16\_years =} \FunctionTok{factor}\NormalTok{(Age\_less\_16\_years),}
    \AttributeTok{Outcome =} \FunctionTok{factor}\NormalTok{(Outcome)}
\NormalTok{  )}

\FunctionTok{write.csv}\NormalTok{(dados, }\StringTok{"dados\_tratados.csv"}\NormalTok{)}

\CommentTok{\#{-}{-}{-}{-}{-}{-}{-}{-}{-}{-}{-}{-}{-}{-}{-} Análise de teste diagnósticos {-}{-}{-}{-}{-}{-}{-}{-}{-}{-}{-}{-}{-}{-}{-}{-}{-}{-}{-}{-}{-}{-}{-}}
\CommentTok{\# A função confusionMatrix retorna:}
\CommentTok{\# Tabela teste x resultado real}
\CommentTok{\# Acurácia}
\CommentTok{\# Kappa (análise de concordância)}
\CommentTok{\# Teste de Mcnemar}
\CommentTok{\# Sensibilidade}
\CommentTok{\# Especificidade}

\CommentTok{\# métricas para NS1}
\FunctionTok{confusionMatrix}\NormalTok{(}\FunctionTok{as.factor}\NormalTok{(dados}\SpecialCharTok{$}\NormalTok{NS1), }\FunctionTok{as.factor}\NormalTok{(dados}\SpecialCharTok{$}\NormalTok{Outcome), }\AttributeTok{positive =} \StringTok{"Positive"}\NormalTok{)}
\CommentTok{\#métricas para IgG}
\FunctionTok{confusionMatrix}\NormalTok{(}\FunctionTok{as.factor}\NormalTok{(dados}\SpecialCharTok{$}\NormalTok{IgG), }\FunctionTok{as.factor}\NormalTok{(dados}\SpecialCharTok{$}\NormalTok{Outcome), }\AttributeTok{positive =} \StringTok{"Positive"}\NormalTok{)}
\CommentTok{\#métricas para IgM}
\FunctionTok{confusionMatrix}\NormalTok{(}\FunctionTok{as.factor}\NormalTok{(dados}\SpecialCharTok{$}\NormalTok{IgM), }\FunctionTok{as.factor}\NormalTok{(dados}\SpecialCharTok{$}\NormalTok{Outcome), }\AttributeTok{positive =} \StringTok{"Positive"}\NormalTok{)}

\CommentTok{\#{-}{-}{-}{-}{-}{-}{-}{-}{-}{-}{-}{-}{-}{-}{-} Diferença de duas proporções {-}{-}{-}{-}{-}{-}{-}{-}{-}{-}{-}{-}{-}{-}{-}{-}{-}{-}{-}{-}{-}{-}{-}}


\NormalTok{comparar\_proporcoes }\OtherTok{\textless{}{-}} \ControlFlowTok{function}\NormalTok{(data, grupo\_var, desfecho\_var, }\AttributeTok{positivo\_label =} \StringTok{"Positive"}\NormalTok{, }\AttributeTok{conf.level =} \FloatTok{0.95}\NormalTok{) \{}
  \CommentTok{\# Extrair os dois níveis do grupo}
\NormalTok{  grupo\_niveis }\OtherTok{\textless{}{-}} \FunctionTok{unique}\NormalTok{(data[[grupo\_var]])}
  \ControlFlowTok{if}\NormalTok{ (}\FunctionTok{length}\NormalTok{(grupo\_niveis) }\SpecialCharTok{!=} \DecValTok{2}\NormalTok{) }\FunctionTok{stop}\NormalTok{(}\StringTok{"A variável do grupo deve ter exatamente dois níveis."}\NormalTok{)}
  
\NormalTok{  g1 }\OtherTok{\textless{}{-}}\NormalTok{ grupo\_niveis[}\DecValTok{1}\NormalTok{]}
\NormalTok{  g2 }\OtherTok{\textless{}{-}}\NormalTok{ grupo\_niveis[}\DecValTok{2}\NormalTok{]}
  
  \CommentTok{\# Totais}
\NormalTok{  n1 }\OtherTok{\textless{}{-}} \FunctionTok{nrow}\NormalTok{(dplyr}\SpecialCharTok{::}\FunctionTok{filter}\NormalTok{(data, }\SpecialCharTok{!!}\NormalTok{rlang}\SpecialCharTok{::}\FunctionTok{sym}\NormalTok{(grupo\_var) }\SpecialCharTok{==}\NormalTok{ g1))}
\NormalTok{  n2 }\OtherTok{\textless{}{-}} \FunctionTok{nrow}\NormalTok{(dplyr}\SpecialCharTok{::}\FunctionTok{filter}\NormalTok{(data, }\SpecialCharTok{!!}\NormalTok{rlang}\SpecialCharTok{::}\FunctionTok{sym}\NormalTok{(grupo\_var) }\SpecialCharTok{==}\NormalTok{ g2))}
  
  \CommentTok{\# Positivos}
\NormalTok{  x1 }\OtherTok{\textless{}{-}} \FunctionTok{nrow}\NormalTok{(dplyr}\SpecialCharTok{::}\FunctionTok{filter}\NormalTok{(data, }\SpecialCharTok{!!}\NormalTok{rlang}\SpecialCharTok{::}\FunctionTok{sym}\NormalTok{(grupo\_var) }\SpecialCharTok{==}\NormalTok{ g1, }\SpecialCharTok{!!}\NormalTok{rlang}\SpecialCharTok{::}\FunctionTok{sym}\NormalTok{(desfecho\_var) }\SpecialCharTok{==}\NormalTok{ positivo\_label))}
\NormalTok{  x2 }\OtherTok{\textless{}{-}} \FunctionTok{nrow}\NormalTok{(dplyr}\SpecialCharTok{::}\FunctionTok{filter}\NormalTok{(data, }\SpecialCharTok{!!}\NormalTok{rlang}\SpecialCharTok{::}\FunctionTok{sym}\NormalTok{(grupo\_var) }\SpecialCharTok{==}\NormalTok{ g2, }\SpecialCharTok{!!}\NormalTok{rlang}\SpecialCharTok{::}\FunctionTok{sym}\NormalTok{(desfecho\_var) }\SpecialCharTok{==}\NormalTok{ positivo\_label))}
  
  \CommentTok{\# Proporções}
\NormalTok{  p1 }\OtherTok{\textless{}{-}}\NormalTok{ x1 }\SpecialCharTok{/}\NormalTok{ n1}
\NormalTok{  p2 }\OtherTok{\textless{}{-}}\NormalTok{ x2 }\SpecialCharTok{/}\NormalTok{ n2}
\NormalTok{  diff }\OtherTok{\textless{}{-}}\NormalTok{ p1 }\SpecialCharTok{{-}}\NormalTok{ p2}
  
  \CommentTok{\# Erro padrão e IC}
\NormalTok{  z }\OtherTok{\textless{}{-}} \FunctionTok{qnorm}\NormalTok{(}\DecValTok{1} \SpecialCharTok{{-}}\NormalTok{ (}\DecValTok{1} \SpecialCharTok{{-}}\NormalTok{ conf.level)}\SpecialCharTok{/}\DecValTok{2}\NormalTok{)}
\NormalTok{  se }\OtherTok{\textless{}{-}} \FunctionTok{sqrt}\NormalTok{( (p1 }\SpecialCharTok{*}\NormalTok{ (}\DecValTok{1} \SpecialCharTok{{-}}\NormalTok{ p1)) }\SpecialCharTok{/}\NormalTok{ n1 }\SpecialCharTok{+}\NormalTok{ (p2 }\SpecialCharTok{*}\NormalTok{ (}\DecValTok{1} \SpecialCharTok{{-}}\NormalTok{ p2)) }\SpecialCharTok{/}\NormalTok{ n2 )}
\NormalTok{  lower }\OtherTok{\textless{}{-}}\NormalTok{ diff }\SpecialCharTok{{-}}\NormalTok{ z }\SpecialCharTok{*}\NormalTok{ se}
\NormalTok{  upper }\OtherTok{\textless{}{-}}\NormalTok{ diff }\SpecialCharTok{+}\NormalTok{ z }\SpecialCharTok{*}\NormalTok{ se}
  
  \CommentTok{\# Retorno}
\NormalTok{  result }\OtherTok{\textless{}{-}} \FunctionTok{list}\NormalTok{(}
    \AttributeTok{grupo\_1 =}\NormalTok{ g1,}
    \AttributeTok{grupo\_2 =}\NormalTok{ g2,}
    \AttributeTok{total\_g1 =}\NormalTok{ n1,}
    \AttributeTok{total\_g2 =}\NormalTok{ n2,}
    \AttributeTok{positivos\_g1 =}\NormalTok{ x1,}
    \AttributeTok{positivos\_g2 =}\NormalTok{ x2,}
    \AttributeTok{prop\_g1 =}\NormalTok{ p1,}
    \AttributeTok{prop\_g2 =}\NormalTok{ p2,}
    \AttributeTok{diff\_prop =}\NormalTok{ diff,}
    \AttributeTok{ic\_95 =} \FunctionTok{c}\NormalTok{(lower, upper)}
\NormalTok{  )}
  
  \FunctionTok{return}\NormalTok{(result)}
\NormalTok{\}}

\CommentTok{\# Para Gender}
\NormalTok{res\_gender }\OtherTok{\textless{}{-}} \FunctionTok{comparar\_proporcoes}\NormalTok{(dados, }\AttributeTok{grupo\_var =} \StringTok{"Gender"}\NormalTok{, }\AttributeTok{desfecho\_var =} \StringTok{"Outcome"}\NormalTok{)}
\FunctionTok{print}\NormalTok{(res\_gender)}

\CommentTok{\# Para AreaType}
\NormalTok{res\_areatype }\OtherTok{\textless{}{-}} \FunctionTok{comparar\_proporcoes}\NormalTok{(dados, }\AttributeTok{grupo\_var =} \StringTok{"AreaType"}\NormalTok{, }\AttributeTok{desfecho\_var =} \StringTok{"Outcome"}\NormalTok{)}
\FunctionTok{print}\NormalTok{(res\_areatype)}

\CommentTok{\# Para Age\_less\_16\_years}
\NormalTok{res\_age }\OtherTok{\textless{}{-}} \FunctionTok{comparar\_proporcoes}\NormalTok{(dados, }\AttributeTok{grupo\_var =} \StringTok{"Age\_less\_16\_years"}\NormalTok{, }\AttributeTok{desfecho\_var =} \StringTok{"Outcome"}\NormalTok{)}
\FunctionTok{print}\NormalTok{(res\_age)}

\CommentTok{\#{-}{-}{-}{-}{-}{-}{-}{-}{-}{-}{-}{-}{-}{-}{-} Análises de Razão de Chances {-}{-}{-}{-}{-}{-}{-}{-}{-}{-}{-}{-}{-}{-}{-}{-}{-}{-}{-}{-}{-}{-}{-}}

\DocumentationTok{\#\#\#\# Gender}

\CommentTok{\# Contagens}
\NormalTok{a }\OtherTok{\textless{}{-}} \FunctionTok{nrow}\NormalTok{(}\FunctionTok{filter}\NormalTok{(dados, Gender }\SpecialCharTok{==} \StringTok{"Male"} \SpecialCharTok{\&}\NormalTok{ Outcome }\SpecialCharTok{==} \StringTok{"Positive"}\NormalTok{))}
\NormalTok{b }\OtherTok{\textless{}{-}} \FunctionTok{nrow}\NormalTok{(}\FunctionTok{filter}\NormalTok{(dados, Gender }\SpecialCharTok{==} \StringTok{"Male"} \SpecialCharTok{\&}\NormalTok{ Outcome }\SpecialCharTok{==} \StringTok{"Negative"}\NormalTok{))}
\NormalTok{c }\OtherTok{\textless{}{-}} \FunctionTok{nrow}\NormalTok{(}\FunctionTok{filter}\NormalTok{(dados, Gender }\SpecialCharTok{==} \StringTok{"Female"} \SpecialCharTok{\&}\NormalTok{ Outcome }\SpecialCharTok{==} \StringTok{"Positive"}\NormalTok{))}
\NormalTok{d }\OtherTok{\textless{}{-}} \FunctionTok{nrow}\NormalTok{(}\FunctionTok{filter}\NormalTok{(dados, Gender }\SpecialCharTok{==} \StringTok{"Female"} \SpecialCharTok{\&}\NormalTok{ Outcome }\SpecialCharTok{==} \StringTok{"Negative"}\NormalTok{))}

\CommentTok{\# OR, log(OR), SE}
\NormalTok{or\_gender }\OtherTok{\textless{}{-}}\NormalTok{ (a }\SpecialCharTok{*}\NormalTok{ d) }\SpecialCharTok{/}\NormalTok{ (b }\SpecialCharTok{*}\NormalTok{ c)}
\NormalTok{log\_or\_gender }\OtherTok{\textless{}{-}} \FunctionTok{log}\NormalTok{(or\_gender)}
\NormalTok{se\_log\_or\_gender }\OtherTok{\textless{}{-}} \FunctionTok{sqrt}\NormalTok{(}\DecValTok{1}\SpecialCharTok{/}\NormalTok{a }\SpecialCharTok{+} \DecValTok{1}\SpecialCharTok{/}\NormalTok{b }\SpecialCharTok{+} \DecValTok{1}\SpecialCharTok{/}\NormalTok{c }\SpecialCharTok{+} \DecValTok{1}\SpecialCharTok{/}\NormalTok{d)}

\CommentTok{\# IC 95\%}
\NormalTok{z }\OtherTok{\textless{}{-}} \FunctionTok{qnorm}\NormalTok{(}\FloatTok{0.975}\NormalTok{)}
\NormalTok{ic\_log\_gender }\OtherTok{\textless{}{-}}\NormalTok{ log\_or\_gender }\SpecialCharTok{+} \FunctionTok{c}\NormalTok{(}\SpecialCharTok{{-}}\DecValTok{1}\NormalTok{, }\DecValTok{1}\NormalTok{) }\SpecialCharTok{*}\NormalTok{ z }\SpecialCharTok{*}\NormalTok{ se\_log\_or\_gender}
\NormalTok{ic\_gender }\OtherTok{\textless{}{-}} \FunctionTok{exp}\NormalTok{(ic\_log\_gender)}

\CommentTok{\# Resultado}
\FunctionTok{cat}\NormalTok{(}\StringTok{"🔹 Gender:}\SpecialCharTok{\textbackslash{}n}\StringTok{"}\NormalTok{)}
\FunctionTok{cat}\NormalTok{(}\StringTok{"  OR ="}\NormalTok{, }\FunctionTok{round}\NormalTok{(or\_gender, }\DecValTok{3}\NormalTok{), }\StringTok{"}\SpecialCharTok{\textbackslash{}n}\StringTok{"}\NormalTok{)}
\FunctionTok{cat}\NormalTok{(}\StringTok{"  log(OR) ="}\NormalTok{, }\FunctionTok{round}\NormalTok{(log\_or\_gender, }\DecValTok{3}\NormalTok{), }\StringTok{"}\SpecialCharTok{\textbackslash{}n}\StringTok{"}\NormalTok{)}
\FunctionTok{cat}\NormalTok{(}\StringTok{"  IC 95\% OR = ["}\NormalTok{, }\FunctionTok{round}\NormalTok{(ic\_gender[}\DecValTok{1}\NormalTok{], }\DecValTok{3}\NormalTok{), }\StringTok{","}\NormalTok{, }\FunctionTok{round}\NormalTok{(ic\_gender[}\DecValTok{2}\NormalTok{], }\DecValTok{3}\NormalTok{), }\StringTok{"]}\SpecialCharTok{\textbackslash{}n\textbackslash{}n}\StringTok{"}\NormalTok{)}


\DocumentationTok{\#\#\#\# AreaType}

\NormalTok{a }\OtherTok{\textless{}{-}} \FunctionTok{nrow}\NormalTok{(}\FunctionTok{filter}\NormalTok{(dados, AreaType }\SpecialCharTok{==} \StringTok{"Developed"} \SpecialCharTok{\&}\NormalTok{ Outcome }\SpecialCharTok{==} \StringTok{"Positive"}\NormalTok{))}
\NormalTok{b }\OtherTok{\textless{}{-}} \FunctionTok{nrow}\NormalTok{(}\FunctionTok{filter}\NormalTok{(dados, AreaType }\SpecialCharTok{==} \StringTok{"Developed"} \SpecialCharTok{\&}\NormalTok{ Outcome }\SpecialCharTok{==} \StringTok{"Negative"}\NormalTok{))}
\NormalTok{c }\OtherTok{\textless{}{-}} \FunctionTok{nrow}\NormalTok{(}\FunctionTok{filter}\NormalTok{(dados, AreaType }\SpecialCharTok{==} \StringTok{"Undeveloped"} \SpecialCharTok{\&}\NormalTok{ Outcome }\SpecialCharTok{==} \StringTok{"Positive"}\NormalTok{))}
\NormalTok{d }\OtherTok{\textless{}{-}} \FunctionTok{nrow}\NormalTok{(}\FunctionTok{filter}\NormalTok{(dados, AreaType }\SpecialCharTok{==} \StringTok{"Undeveloped"} \SpecialCharTok{\&}\NormalTok{ Outcome }\SpecialCharTok{==} \StringTok{"Negative"}\NormalTok{))}

\NormalTok{or\_area }\OtherTok{\textless{}{-}}\NormalTok{ (a }\SpecialCharTok{*}\NormalTok{ d) }\SpecialCharTok{/}\NormalTok{ (b }\SpecialCharTok{*}\NormalTok{ c)}
\NormalTok{log\_or\_area }\OtherTok{\textless{}{-}} \FunctionTok{log}\NormalTok{(or\_area)}
\NormalTok{se\_log\_or\_area }\OtherTok{\textless{}{-}} \FunctionTok{sqrt}\NormalTok{(}\DecValTok{1}\SpecialCharTok{/}\NormalTok{a }\SpecialCharTok{+} \DecValTok{1}\SpecialCharTok{/}\NormalTok{b }\SpecialCharTok{+} \DecValTok{1}\SpecialCharTok{/}\NormalTok{c }\SpecialCharTok{+} \DecValTok{1}\SpecialCharTok{/}\NormalTok{d)}

\NormalTok{ic\_log\_area }\OtherTok{\textless{}{-}}\NormalTok{ log\_or\_area }\SpecialCharTok{+} \FunctionTok{c}\NormalTok{(}\SpecialCharTok{{-}}\DecValTok{1}\NormalTok{, }\DecValTok{1}\NormalTok{) }\SpecialCharTok{*}\NormalTok{ z }\SpecialCharTok{*}\NormalTok{ se\_log\_or\_area}
\NormalTok{ic\_area }\OtherTok{\textless{}{-}} \FunctionTok{exp}\NormalTok{(ic\_log\_area)}

\FunctionTok{cat}\NormalTok{(}\StringTok{"🔹 AreaType:}\SpecialCharTok{\textbackslash{}n}\StringTok{"}\NormalTok{)}
\FunctionTok{cat}\NormalTok{(}\StringTok{"  OR ="}\NormalTok{, }\FunctionTok{round}\NormalTok{(or\_area, }\DecValTok{3}\NormalTok{), }\StringTok{"}\SpecialCharTok{\textbackslash{}n}\StringTok{"}\NormalTok{)}
\FunctionTok{cat}\NormalTok{(}\StringTok{"  log(OR) ="}\NormalTok{, }\FunctionTok{round}\NormalTok{(log\_or\_area, }\DecValTok{3}\NormalTok{), }\StringTok{"}\SpecialCharTok{\textbackslash{}n}\StringTok{"}\NormalTok{)}
\FunctionTok{cat}\NormalTok{(}\StringTok{"  IC 95\% OR = ["}\NormalTok{, }\FunctionTok{round}\NormalTok{(ic\_area[}\DecValTok{1}\NormalTok{], }\DecValTok{3}\NormalTok{), }\StringTok{","}\NormalTok{, }\FunctionTok{round}\NormalTok{(ic\_area[}\DecValTok{2}\NormalTok{], }\DecValTok{3}\NormalTok{), }\StringTok{"]}\SpecialCharTok{\textbackslash{}n\textbackslash{}n}\StringTok{"}\NormalTok{)}

\DocumentationTok{\#\#\#\# Age\_less\_16\_years }

\NormalTok{a }\OtherTok{\textless{}{-}} \FunctionTok{nrow}\NormalTok{(}\FunctionTok{filter}\NormalTok{(dados, Age\_less\_16\_years }\SpecialCharTok{==} \StringTok{"Yes"} \SpecialCharTok{\&}\NormalTok{ Outcome }\SpecialCharTok{==} \StringTok{"Positive"}\NormalTok{))}
\NormalTok{b }\OtherTok{\textless{}{-}} \FunctionTok{nrow}\NormalTok{(}\FunctionTok{filter}\NormalTok{(dados, Age\_less\_16\_years }\SpecialCharTok{==} \StringTok{"Yes"} \SpecialCharTok{\&}\NormalTok{ Outcome }\SpecialCharTok{==} \StringTok{"Negative"}\NormalTok{))}
\NormalTok{c }\OtherTok{\textless{}{-}} \FunctionTok{nrow}\NormalTok{(}\FunctionTok{filter}\NormalTok{(dados, Age\_less\_16\_years }\SpecialCharTok{==} \StringTok{"No"} \SpecialCharTok{\&}\NormalTok{ Outcome }\SpecialCharTok{==} \StringTok{"Positive"}\NormalTok{))}
\NormalTok{d }\OtherTok{\textless{}{-}} \FunctionTok{nrow}\NormalTok{(}\FunctionTok{filter}\NormalTok{(dados, Age\_less\_16\_years }\SpecialCharTok{==} \StringTok{"No"} \SpecialCharTok{\&}\NormalTok{ Outcome }\SpecialCharTok{==} \StringTok{"Negative"}\NormalTok{))}

\NormalTok{or\_age }\OtherTok{\textless{}{-}}\NormalTok{ (a }\SpecialCharTok{*}\NormalTok{ d) }\SpecialCharTok{/}\NormalTok{ (b }\SpecialCharTok{*}\NormalTok{ c)}
\NormalTok{log\_or\_age }\OtherTok{\textless{}{-}} \FunctionTok{log}\NormalTok{(or\_age)}
\NormalTok{se\_log\_or\_age }\OtherTok{\textless{}{-}} \FunctionTok{sqrt}\NormalTok{(}\DecValTok{1}\SpecialCharTok{/}\NormalTok{a }\SpecialCharTok{+} \DecValTok{1}\SpecialCharTok{/}\NormalTok{b }\SpecialCharTok{+} \DecValTok{1}\SpecialCharTok{/}\NormalTok{c }\SpecialCharTok{+} \DecValTok{1}\SpecialCharTok{/}\NormalTok{d)}

\NormalTok{ic\_log\_age }\OtherTok{\textless{}{-}}\NormalTok{ log\_or\_age }\SpecialCharTok{+} \FunctionTok{c}\NormalTok{(}\SpecialCharTok{{-}}\DecValTok{1}\NormalTok{, }\DecValTok{1}\NormalTok{) }\SpecialCharTok{*}\NormalTok{ z }\SpecialCharTok{*}\NormalTok{ se\_log\_or\_age}
\NormalTok{ic\_age }\OtherTok{\textless{}{-}} \FunctionTok{exp}\NormalTok{(ic\_log\_age)}

\FunctionTok{cat}\NormalTok{(}\StringTok{"🔹 Age \textless{} 16:}\SpecialCharTok{\textbackslash{}n}\StringTok{"}\NormalTok{)}
\FunctionTok{cat}\NormalTok{(}\StringTok{"  OR ="}\NormalTok{, }\FunctionTok{round}\NormalTok{(or\_age, }\DecValTok{3}\NormalTok{), }\StringTok{"}\SpecialCharTok{\textbackslash{}n}\StringTok{"}\NormalTok{)}
\FunctionTok{cat}\NormalTok{(}\StringTok{"  log(OR) ="}\NormalTok{, }\FunctionTok{round}\NormalTok{(log\_or\_age, }\DecValTok{3}\NormalTok{), }\StringTok{"}\SpecialCharTok{\textbackslash{}n}\StringTok{"}\NormalTok{)}
\FunctionTok{cat}\NormalTok{(}\StringTok{"  IC 95\% OR = ["}\NormalTok{, }\FunctionTok{round}\NormalTok{(ic\_age[}\DecValTok{1}\NormalTok{], }\DecValTok{3}\NormalTok{), }\StringTok{","}\NormalTok{, }\FunctionTok{round}\NormalTok{(ic\_age[}\DecValTok{2}\NormalTok{], }\DecValTok{3}\NormalTok{), }\StringTok{"]}\SpecialCharTok{\textbackslash{}n\textbackslash{}n}\StringTok{"}\NormalTok{)}

\CommentTok{\# {-}{-}{-}{-}{-}{-}{-}{-}{-}{-}{-}{-}{-} Modelo de Regressão Logística}
\end{Highlighting}
\end{Shaded}


\end{document}
